\hypertarget{linear-regression-and-machine-learning}{%
\section{09. Linear regression and machine
learning}\label{linear-regression-and-machine-learning}}

\hypertarget{last-time}{%
\subsection{Last time}\label{last-time}}

\begin{itemize}
\item
  Exact enumeration for first-passage times
\item
  Continuous random variables
\item
  Probability density function
\item
  Central Limit Theorem
\end{itemize}

\hypertarget{goal-for-today}{%
\subsection{Goal for today}\label{goal-for-today}}

\begin{itemize}
\item
  What is machine learning?
\item
  Linear regression
\item
  Derivatives and how to calculate them
\end{itemize}

\hypertarget{understanding-data}{%
\subsection{Understanding data}\label{understanding-data}}

\begin{itemize}
\item
  Suppose have \textbf{data} from some experiment / process
\item
  Examples:

  \begin{itemize}
  \item
    Stock price as function of time
  \item
    Jet noise as a function of air flow speed
  \item
    Sales as function of advertising budget
  \item
    Length of spring as function of force applied
  \item
    Variance of random walker as function of time
  \end{itemize}
\end{itemize}

\hypertarget{characteristics-of-data}{%
\subsection{Characteristics of data}\label{characteristics-of-data}}

\begin{itemize}
\item
  Data have \textbf{inputs}: we specify or measure them
\item
  Data have \textbf{outputs}: we are interested in how they change when
  inputs change
\item
  Data are \textbf{noisy}: intrinsic random fluctuations
\end{itemize}

\hypertarget{understanding-data-ii-models}{%
\subsection{Understanding data II:
Models}\label{understanding-data-ii-models}}

\begin{itemize}
\item
  Want to \textbf{understand} / characterise structure of data: world of
  \textbf{statistics}
\item
  Also want to \textbf{predict} ``response'' for new input data: world
  of \textbf{machine learning}
\item
  To understand data we will impose some kind of \textbf{structure} on
  it -- a \textbf{model}
\item
  Model describes relationship that we think data has
\end{itemize}

\hypertarget{linear-regression}{%
\subsection{Linear regression}\label{linear-regression}}

\begin{itemize}
\item
  One of simplest models: \textbf{linear regression} (bad name)
\item
  Relates quantitative measurements
\item
  Assumes linear (affine) relationship between inputs \(X\) and outputs
  \(Y\):
\end{itemize}

\[Y = f(X) = aX + b \quad + \epsilon\]

\begin{itemize}
\item
  \(\epsilon\) describes noise / fluctuations
\item
  Unknown \textbf{parameters} \(a\) and \(b\)
\end{itemize}

\hypertarget{machine-learning-fitting}{%
\subsection{Machine learning: Fitting}\label{machine-learning-fitting}}

\begin{itemize}
\item
  Notation: Data \((x_i, y_i)\)
\item
  Input \(x_i\) and corresponding output \(y_i\), for \(i=1,\ldots, N\)
\item
  Given data we want to \textbf{learn} parameters \(a\) and \(b\) in
  model
\item
  \textbf{Learning} is just \textbf{fitting}!
\item
  What is fitting?
\end{itemize}

\hypertarget{fitting}{%
\subsection{Fitting}\label{fitting}}

\begin{itemize}
\item
  Fitting: Find parameters \(a\) and \(b\) in model that \textbf{best}
  describe data
\item
  What does \textbf{best} mean?
\item
  Need way to decide what is ``best'' fit
\item
  Need to \textbf{minimize} a \textbf{cost} / \textbf{loss} function
  \(L\)
\item
  How choose loss function?
\end{itemize}

\hypertarget{least-squares}{%
\subsection{Least squares}\label{least-squares}}

\begin{itemize}
\item
  Common solution: \textbf{Least squares}
\item
  Sum of squares of distances of data from line
\item
  This is a function \(L(a, b)\) of parameters \(a\) and \(b\)
\item
  Find values of \(a\) and \(b\) that minimize \(L(a, b)\)
\item
  \textbf{How}?
\end{itemize}

\hypertarget{minimization-in-1d}{%
\subsection{Minimization in 1D}\label{minimization-in-1d}}

\begin{itemize}
\item
  We have a 2D problem. Simplify to 1D
\item
  Think of hill of height \(h(x)\) as function of \(x\)
\item
  How find minimum of hill?
\item
  ``Roll down the hill''
\item
  Take steps and move in direction that \textbf{decreases} \(L\)
\item
  How do we talk about functions \textbf{decreasing}?
\end{itemize}

\hypertarget{reminder-derivatives}{%
\subsection{Reminder: Derivatives}\label{reminder-derivatives}}

\begin{itemize}
\item
  \textbf{Derivative} = \textbf{rate of change}
\item
  Increasing \(\Longleftrightarrow\) derivative \textgreater{} 0
\item
  Decreasing \(\Longleftrightarrow\) derivative 0
\end{itemize}

\hypertarget{derivatives-ii}{%
\subsection{Derivatives II}\label{derivatives-ii}}

\begin{itemize}
\item
  \textbf{Derivative} of function \(f: \mathbb{R} \to \mathbb{R}\) at
  point \(a\) is slope of \textbf{tangent line}
\item
  Notation: \(f'(a)\), or \(\left. \frac{df}{dx} \right| _a\) (if you
  must).
\item
  \textbf{Tangent line} is straight line that ``touches'' graph of
  function at point
\item
  Formal definition of derivative of \(f: \mathbb{R} \to \mathbb{R}\) at
  \(a \in \mathbb{R}\):
  \[f'(a) := \lim_{h \to 0} \frac{f(a + h) - f(a)}{h}\]
\item
  Intuition: Limit of slopes of ``secant lines'' (``rise over run'')
\end{itemize}

\hypertarget{why-do-we-care-about-derivatives}{%
\subsection{Why do we care about
derivatives?}\label{why-do-we-care-about-derivatives}}

\begin{itemize}
\item
  They tell us how function looks ``locally'' (close to a point).
\item
  E.g. used to analyze dynamics near a fixed point.
\item
  Some applications:

  \begin{itemize}
  \tightlist
  \item
    optimization
  \item
    finding roots (zeros)
  \item
    sensitivity: ``how much does output change when input varies''
  \end{itemize}
\end{itemize}

\hypertarget{how-to-calculate-derivatives-numerically}{%
\subsection{\texorpdfstring{How to \emph{calculate} derivatives
numerically}{How to calculate derivatives numerically}}\label{how-to-calculate-derivatives-numerically}}

\begin{itemize}
\item
  Numerically: Cannot take limit \(h \to 0\)
\item
  So don't! Fix a finite, non-zero \(h\) to get \textbf{finite
  difference} approximation:

  \(f'(a) \simeq \frac{f(a + h) - f(a)}{h}\)
\item
  How good is this approximation? PS4
\item
  Can we do better by calculating derivatives exactly?
\end{itemize}

\hypertarget{a-different-point-of-view}{%
\subsection{A different point of view}\label{a-different-point-of-view}}

\begin{itemize}
\item
  Rewrite definition in more useful way:
\item
  Get rid of that annoying limit! (Or, rather, hide it):
\end{itemize}

\[\lim_{h \to 0} \left[ \frac{f(a + h) - f(a)}{h} - f'(a) \right] = 0\]

\begin{itemize}
\item
  Write as \[\frac{f(a + h) - f(a)}{h} - f'(a)  = o(h)\]
\item
  Define \(o(h)\) to mean ``any function \(g(h)\) that satisfies
  \(g(h) / h \to 0\) when \(h \to 0\)''.
\end{itemize}

\hypertarget{section}{%
\subsection{}\label{section}}

\begin{itemize}
\item
  Then \[f(a + h) = f(a) + h f'(a) + o(h).\]
\item
  Conversely: If can find \(A\) and \(B\) with
  \(f(a + h) = A + Bh + o(h)\), then \(A = f(a)\) and \(B = f'(a)\).
\item
  Use this to calculate derivatives!
\item
  \textbf{Intuition}: Tangent line is best affine approximation to \(f\)
  near \(a\).
\end{itemize}

\hypertarget{infinitesimals}{%
\subsection{Infinitesimals}\label{infinitesimals}}

\begin{itemize}
\item
  Simplify by thinking of ``infinitesimal'' perturbation \(\epsilon\)
\item
  With \(\epsilon^2 = 0\)
\item
  So \(f(a + \epsilon) = f(a) + \epsilon f'(a)\)
\item
  Expand \(f(a + \epsilon)\); coefficient of \(\epsilon\) is derivative
\end{itemize}

\hypertarget{sum-rule-for-derivatives}{%
\subsection{Sum rule for derivatives}\label{sum-rule-for-derivatives}}

\begin{itemize}
\item
  Sum of two functions: \[ (f + g)(x) := f(x) + g(x) \]
\item
  Its derivative:
\end{itemize}

\[
[f + g](a + \epsilon) &= f(a + \epsilon) + g(a + \epsilon)  \]
\[ [f(a) + \epsilon f'(a)]  + [g(a) + \epsilon g'(a)] \]
\[ [f(a) + g(a)] + [f'(a) + g'(a)] \epsilon .
\]

\begin{itemize}
\tightlist
\item
  Hence \((f + g)'(a)\) = (coefficient of \(\epsilon\)) =
  \(f'(a) + g'(a)\).
\end{itemize}

\hypertarget{product-rule-for-derivatives}{%
\subsection{Product rule for
derivatives}\label{product-rule-for-derivatives}}

\begin{itemize}
\item
  Product of two functions:

  \[(f \cdot g)(x) := f(x) \cdot g(x)\]

  (Here \(\cdot\) is normal scalar multiplication)
\item
  Its derivative:
\end{itemize}

\[
[f \cdot g](a + \epsilon) &= f(a + \epsilon) \cdot g(a + \epsilon)  \]
\[ [f(a) + \epsilon f'(a)] \cdot [g(a) + \epsilon g'(a)] \]
\[ [f(a) \cdot g(a)] + [f(a) g'(a) + g(a) f'(a)] \epsilon.
\]

\begin{itemize}
\tightlist
\item
  Hence \((f \cdot g)'(a)\) = (coefficient of \(\epsilon\)) =
  \(f(a) g'(a) + g(a) f'(a)\).
\end{itemize}

\hypertarget{derivatives-by-executing-rules-algorithmic-differentiation}{%
\subsection{Derivatives by executing rules: Algorithmic
differentiation}\label{derivatives-by-executing-rules-algorithmic-differentiation}}

\begin{itemize}
\item
  For more complicated function, execute each rule in turn
\item
  E.g. \(h(x) = 3x^2 + 2x\) is \(h(x) = +(3 * (x * x), 2 * x)\)
\item
  Differentiating by hand feels pointless -- we are \emph{executing an
  algorithm}
\item
  Computers are good at that! \textbf{Algorithmic / automatic
  differentiation}
\item
  How \emph{encode} rules to find \(f'(a)\) on computer?
\item
  What information do we need for each function?
\end{itemize}

\hypertarget{information-we-need}{%
\subsection{Information we need}\label{information-we-need}}

\begin{itemize}
\item
  Fix point \(a\) where taking derivatives
\item
  For each function \(f\), need exactly \emph{two} pieces of
  information:
\item
  Value \(f(a)\) and derivative \(f'(a)\).
\item
  So can represent function using just those two pieces of information
\item
  How represent in Julia?
\end{itemize}

\hypertarget{representation-in-julia}{%
\subsection{Representation in Julia}\label{representation-in-julia}}

\begin{itemize}
\item
  Need to group together 2 pieces of information
\item
  Could use tuple or vector etc.
\item
  But want to implement novel \textbf{behaviour}, i.e.~rules for \(+\)
  and \(\times\).
\item
  So instead should \emph{define a new type}
\item
  Commonly called ``dual number''
\end{itemize}

\hypertarget{dual-number-type}{%
\subsection{Dual number type}\label{dual-number-type}}

\begin{itemize}
\item
  Make an immutable dual number type:

\begin{Shaded}
\begin{Highlighting}[]
\NormalTok{struct Dual}
\NormalTok{    value::}\DataTypeTok{Float64}
\NormalTok{    deriv::}\DataTypeTok{Float64}
\KeywordTok{end}
\end{Highlighting}
\end{Shaded}
\item
  Recall: this is template for box holding two variables, \texttt{value}
  and \texttt{derivative}.
\item
  \texttt{Dual(a,\ b)} corresponds directly to \(a + \epsilon b\)
\end{itemize}

\hypertarget{implementing-arithmetic}{%
\subsection{Implementing arithmetic}\label{implementing-arithmetic}}

\begin{itemize}
\item
  To implement arithmetic, import relevant functions:

\begin{Shaded}
\begin{Highlighting}[]
\KeywordTok{import}\NormalTok{ Base: +, *}
\end{Highlighting}
\end{Shaded}
\item
  Add methods acting on objects of type \texttt{Dual}:
  \texttt{julia\ \ \ +(f::Dual,\ g::Dual)\ =\ Dual(f.value\ +\ g.value,\ \ \ \ \ \ \ \ \ \ \ \ \ \ \ \ \ \ \ \ \ \ \ \ \ \ \ \ \ \ \ f.deriv\ +\ g.deriv)}
\item
  Here we have defined the sum of two functions to have the correct
  value and derivative
\end{itemize}

\hypertarget{differentiation}{%
\subsection{Differentiation}\label{differentiation}}

\begin{itemize}
\item
  Suppose have Julia function like \texttt{f(x)\ =\ x\^{}2\ +\ 2x}
\item
  How differentiate \(f\) at \(a = 3\)?
\item
  \(f(a + \epsilon) = f(a) + \epsilon f'(a)\)
\item
  So pass in \(a + \epsilon\) to \(f\), i.e.~\texttt{Dual(a,\ 1)}.
\item
  {[}Represents identity function \(x \mapsto x\) at \(x=a\), with
  derivative \(1\){]}.
\item
  \textbf{Exercise}: Write function \texttt{differentiate} taking
  function \texttt{f} and value \texttt{a} that calculates \(f'(a)\).
\end{itemize}

\hypertarget{review}{%
\subsection{Review}\label{review}}

\begin{itemize}
\item
  Motivation: Linear regression -- fitting straight line
\item
  Optimization wants derivatives
\item
  Calculate derivatives using automatic differentiation
\end{itemize}
