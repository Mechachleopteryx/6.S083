\begin{frame}{13. Networks}
\protect\hypertarget{networks}{}

\begin{block}{Last time}

\begin{itemize}
\item
  Neurons as classifiers
\item
  Neural networks
\item
  Stochastic gradient descent
\end{itemize}

\end{block}

\begin{block}{Goals for today}

\begin{itemize}
\item
  Connectivity: Networks / graphs
\item
  Shortest path
\item
  Random walks
\end{itemize}

\end{block}

\end{frame}

\begin{frame}[fragile]{Modelling connectivity}
\protect\hypertarget{modelling-connectivity}{}

\begin{itemize}
\item
  Real-world systems often depend on \textbf{connectivity}:

  \begin{itemize}
  \tightlist
  \item
    Travel: Massachusetts road network; world airports
  \item
    Social network: Alumni of MIT and their friendships (or not)
  \item
    Network of interacting genes turning one another on and off
  \end{itemize}
\item
  Model as \textbf{networks} / \textbf{graphs}
\end{itemize}

\begin{block}{Networks generalize themes of course}

\begin{itemize}
\item
  Dynamics of genetic networks
\item
  Contact network: trace people / animals using GPS as they move
\item
  How does an epidemic spread via plane travel
\item
  Random walk on network: effects of altered gene
\item
  How long until two of them meet?
\end{itemize}

\end{block}

\begin{block}{What is a network?}

\begin{itemize}
\item
  A \textbf{network} is the pair \((V, E)\)
\item
  \(V\) is set of \textbf{vertices} / \textbf{nodes}
  \(i = 1, \ldots, N\)
\item
  \(E\) is set of edges joining them, e.g.~edge \((3, 5)\) joins nodes 3
  and 5
\item
  Tells us how nodes are connected
\item
  \textbf{Directed} (one-way arrows) or \textbf{undirected} (arrows go
  both ways)
\item
  Edges may have \textbf{weights}, e.g.~distance between nodes
\end{itemize}

\end{block}

\begin{block}{What questions would we like to answer?}

\begin{itemize}
\item
  Where can I move from here? (Where am I connected to?)
\item
  Where can I move \emph{from} to get here \emph{from}? (Not necessarily
  the same)
\item
  What is \textbf{degree} = number of neighbours
\item
  Am I connected to a given node?
\item
  What is \textbf{distance} from given node? \(= \infty\) if not
  connected
\end{itemize}

\end{block}

\begin{block}{How can we represent a network?}

\begin{itemize}
\item
  How encode a network in the computer?
\item
  What are possible representations?
\item
  I.e. possible data structures
\end{itemize}

\end{block}

\begin{block}{Possible solutions}

\begin{itemize}
\item
  Vector of edges \((i, j)\)
\item
  Adjacency matrix: \(A_{ij} = 1\) if \(i\) and \(j\) are connected,
  \(0\) otherwise
\item
  Vector of neighbours of each node
\item
  Which is better?
\item
  Depends on how \textbf{sparse} the network / matrix is
\end{itemize}

\end{block}

\begin{block}{Examples of networks:}

\begin{itemize}
\item
  Completely connected
\item
  Square / cubic grid
\item
  \textbf{Erdős--Rényi}: \(N\) nodes, each connected to \(k\) neighbours
  at random
\end{itemize}

\end{block}

\begin{block}{Networks in Julia}

\begin{itemize}
\item
  Simple implementation:
\item
  \texttt{Node} and \texttt{Network} types
\item
  \texttt{add\_vertex!} and \texttt{add\_edge!} methods
\item
  \(\ge 2\) Julia packages: \texttt{LightGraphs.jl} +
  \texttt{MatrixNetworks.jl}
\end{itemize}

\end{block}

\begin{block}{Small-world networks}

\begin{itemize}
\item
  Watts--Strogatz 1998
\item
  Start from ring with \(N\) nodes, each connected to \(k\) neighbours
\item
  Rewire edges at random with probability \(p\), avoiding duplicate
  edges
\item
  What is effect on path length?
\end{itemize}

\end{block}

\begin{block}{Calculating shortest paths}

\begin{itemize}
\item
  Several algorithms to find shortest paths
\item
  Shortest path from single node: Dijkstra algorithm
\item
  All pairs of shortest path lengths: Floyd--Warshall algorithm
\end{itemize}

\end{block}

\begin{block}{Floyd--Warshall}

\begin{itemize}
\item
  Suppose vertices are \((1, \ldots, n)\)
\item
  Consider pair of vertices \(i, j\)
\item
  Look at paths \(i \to j\) with intermediate vertices in
  \((1, \ldots, k)\)
\item
  Call \(p\) a path of minimal length / weight
\item
  Relate to those using vertices \((1, \ldots, k-1)\) only
\end{itemize}

\end{block}

\begin{block}{}

\begin{itemize}
\item
  If \(k\) is not intermediate vertex of \(p\) then \(p\) is in set of
  those using only \((1, \ldots, k)\)
\item
  If \(k\) is intermediate, write \(p\) as \(i \to k \to j\).
\item
  Subpaths are shortest paths
\end{itemize}

\end{block}

\begin{block}{}

\begin{itemize}
\item
  Call \(d_{ij}^k\) the distance of a shortest path from \(i\) to \(j\)
  with intermediates in \((1, \ldots, k)\).
\item
  \(d_{ij}^0 = w_{ij}\), weight / adjacency matrix
\item
  \(d_{ij}^k = \min(d_{ij}^{k-1}, \, d_{ik}^{k-1} + d_{kj}^{k-1}\))
\end{itemize}

\end{block}

\begin{block}{}

\begin{Shaded}
\begin{Highlighting}[]
\KeywordTok{function}\NormalTok{ floyd_warshall(W)}
\NormalTok{    n = size(W, }\FloatTok{1}\NormalTok{)}
\NormalTok{    D = deepcopy(W)}
\NormalTok{    D[D .== }\FloatTok{0}\NormalTok{] .= }\FloatTok{1000}
    \CommentTok{# D[D .== 1] .= 0}

\NormalTok{    Ds = [deepcopy(D)]}

    \KeywordTok{for}\NormalTok{ k = }\FloatTok{1}\NormalTok{:n}
\NormalTok{        new_D = zeros(size(D))}
        \KeywordTok{for}\NormalTok{ i }\KeywordTok{in} \FloatTok{1}\NormalTok{:n, j }\KeywordTok{in} \FloatTok{1}\NormalTok{:n}
            \KeywordTok{if}\NormalTok{ i == j}
\NormalTok{                new_D[i, j] = }\FloatTok{0}
                \KeywordTok{continue}
            \KeywordTok{end}
\NormalTok{            new_D[i, j] = min(D[i, j], D[i, k] + D[k, j])}
        \KeywordTok{end}

\NormalTok{        push!(Ds, deepcopy(new_D))}
\NormalTok{        D = new_D}
    \KeywordTok{end}

    \KeywordTok{return}\NormalTok{ Ds}
\KeywordTok{end}

\NormalTok{floyd_warshall([}\FloatTok{0} \FloatTok{1} \FloatTok{0}\NormalTok{;}
                \FloatTok{1} \FloatTok{0} \FloatTok{1}\NormalTok{;}
                \FloatTok{0} \FloatTok{1} \FloatTok{0}\NormalTok{])}

\end{Highlighting}
\end{Shaded}

\end{block}

\begin{block}{Small-world property}

\begin{itemize}
\item
  Distance between 2 arbitrary nodes is ``small'':
\item
  How fast does it grow with
\item
  Cf. conversation at a party with somebody you don't know -- ``six
  degrees of separation''
\end{itemize}

\end{block}

\begin{block}{Preferential attachment:}

\begin{itemize}
\item
  Barabási--Albert 1999 (also previously): \textbf{Scale-free} networks
\item
  Model for structure of World Wide Web / internet
\item
  Suppose node \(i\) has degree (number of neighbours) \(d_i\)
\item
  At each step, choose a node \(i\) with probability \(\propto d_i\)
\item
  Attach 1 new node there
\item
  How choose nodes with probability in this way?
\end{itemize}

\end{block}

\begin{block}{Implementation}

\begin{itemize}
\item
  Choose random integer in range between \(1\) and \(\sum_i d_i\)
\item
  Create \emph{cumulative distribution} vector:
\item
  \(c_j := \sum_{i=1}^j d_i\)
\item
  \texttt{cumsum} in Julia
\item
  Search in sorted vector \(c_j\) using bisection search
\item
  \texttt{searchsorted} in Julia
\end{itemize}

\end{block}

\begin{block}{Code}

\begin{verbatim}
```julia
function proportional_choice(d::Vector{Int})
    c = cumsum(d)
    r = rand(1:c[end])

    return searchsorted(c, r)[1]
end
```
\end{verbatim}

\end{block}

\begin{block}{Random walk on a network}

\begin{itemize}
\item
  If at node \(i\), choose one neighbour uniformly and jump there
\item
  Where does the walker spend more time?
\end{itemize}

\end{block}

\begin{block}{Review}

\begin{itemize}
\item
  Networks / graphs
\item
  Data structures for implementation
\item
  Models for social networks: small-world and scale-free
\end{itemize}

\end{block}

\end{frame}
